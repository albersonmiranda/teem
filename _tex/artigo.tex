% Options for packages loaded elsewhere
\PassOptionsToPackage{unicode}{hyperref}
\PassOptionsToPackage{hyphens}{url}
\PassOptionsToPackage{dvipsnames,svgnames,x11names}{xcolor}
%
\documentclass[
  letterpaper,
  DIV=11,
  numbers=noendperiod]{scrreprt}

\usepackage{amsmath,amssymb}
\usepackage{iftex}
\ifPDFTeX
  \usepackage[T1]{fontenc}
  \usepackage[utf8]{inputenc}
  \usepackage{textcomp} % provide euro and other symbols
\else % if luatex or xetex
  \usepackage{unicode-math}
  \defaultfontfeatures{Scale=MatchLowercase}
  \defaultfontfeatures[\rmfamily]{Ligatures=TeX,Scale=1}
\fi
\usepackage{lmodern}
\ifPDFTeX\else  
    % xetex/luatex font selection
\fi
% Use upquote if available, for straight quotes in verbatim environments
\IfFileExists{upquote.sty}{\usepackage{upquote}}{}
\IfFileExists{microtype.sty}{% use microtype if available
  \usepackage[]{microtype}
  \UseMicrotypeSet[protrusion]{basicmath} % disable protrusion for tt fonts
}{}
\makeatletter
\@ifundefined{KOMAClassName}{% if non-KOMA class
  \IfFileExists{parskip.sty}{%
    \usepackage{parskip}
  }{% else
    \setlength{\parindent}{0pt}
    \setlength{\parskip}{6pt plus 2pt minus 1pt}}
}{% if KOMA class
  \KOMAoptions{parskip=half}}
\makeatother
\usepackage{xcolor}
\setlength{\emergencystretch}{3em} % prevent overfull lines
\setcounter{secnumdepth}{5}
% Make \paragraph and \subparagraph free-standing
\ifx\paragraph\undefined\else
  \let\oldparagraph\paragraph
  \renewcommand{\paragraph}[1]{\oldparagraph{#1}\mbox{}}
\fi
\ifx\subparagraph\undefined\else
  \let\oldsubparagraph\subparagraph
  \renewcommand{\subparagraph}[1]{\oldsubparagraph{#1}\mbox{}}
\fi


\providecommand{\tightlist}{%
  \setlength{\itemsep}{0pt}\setlength{\parskip}{0pt}}\usepackage{longtable,booktabs,array}
\usepackage{calc} % for calculating minipage widths
% Correct order of tables after \paragraph or \subparagraph
\usepackage{etoolbox}
\makeatletter
\patchcmd\longtable{\par}{\if@noskipsec\mbox{}\fi\par}{}{}
\makeatother
% Allow footnotes in longtable head/foot
\IfFileExists{footnotehyper.sty}{\usepackage{footnotehyper}}{\usepackage{footnote}}
\makesavenoteenv{longtable}
\usepackage{graphicx}
\makeatletter
\def\maxwidth{\ifdim\Gin@nat@width>\linewidth\linewidth\else\Gin@nat@width\fi}
\def\maxheight{\ifdim\Gin@nat@height>\textheight\textheight\else\Gin@nat@height\fi}
\makeatother
% Scale images if necessary, so that they will not overflow the page
% margins by default, and it is still possible to overwrite the defaults
% using explicit options in \includegraphics[width, height, ...]{}
\setkeys{Gin}{width=\maxwidth,height=\maxheight,keepaspectratio}
% Set default figure placement to htbp
\makeatletter
\def\fps@figure{htbp}
\makeatother
% definitions for citeproc citations
\NewDocumentCommand\citeproctext{}{}
\NewDocumentCommand\citeproc{mm}{%
  \begingroup\def\citeproctext{#2}\cite{#1}\endgroup}
\makeatletter
 % allow citations to break across lines
 \let\@cite@ofmt\@firstofone
 % avoid brackets around text for \cite:
 \def\@biblabel#1{}
 \def\@cite#1#2{{#1\if@tempswa , #2\fi}}
\makeatother
\newlength{\cslhangindent}
\setlength{\cslhangindent}{1.5em}
\newlength{\csllabelwidth}
\setlength{\csllabelwidth}{3em}
\newenvironment{CSLReferences}[2] % #1 hanging-indent, #2 entry-spacing
 {\begin{list}{}{%
  \setlength{\itemindent}{0pt}
  \setlength{\leftmargin}{0pt}
  \setlength{\parsep}{0pt}
  % turn on hanging indent if param 1 is 1
  \ifodd #1
   \setlength{\leftmargin}{\cslhangindent}
   \setlength{\itemindent}{-1\cslhangindent}
  \fi
  % set entry spacing
  \setlength{\itemsep}{#2\baselineskip}}}
 {\end{list}}
\usepackage{calc}
\newcommand{\CSLBlock}[1]{\hfill\break\parbox[t]{\linewidth}{\strut\ignorespaces#1\strut}}
\newcommand{\CSLLeftMargin}[1]{\parbox[t]{\csllabelwidth}{\strut#1\strut}}
\newcommand{\CSLRightInline}[1]{\parbox[t]{\linewidth - \csllabelwidth}{\strut#1\strut}}
\newcommand{\CSLIndent}[1]{\hspace{\cslhangindent}#1}

\KOMAoption{captions}{tableheading}
\makeatletter
\@ifpackageloaded{caption}{}{\usepackage{caption}}
\AtBeginDocument{%
\ifdefined\contentsname
  \renewcommand*\contentsname{Índice}
\else
  \newcommand\contentsname{Índice}
\fi
\ifdefined\listfigurename
  \renewcommand*\listfigurename{Lista de Figuras}
\else
  \newcommand\listfigurename{Lista de Figuras}
\fi
\ifdefined\listtablename
  \renewcommand*\listtablename{Lista de Tabelas}
\else
  \newcommand\listtablename{Lista de Tabelas}
\fi
\ifdefined\figurename
  \renewcommand*\figurename{Figura}
\else
  \newcommand\figurename{Figura}
\fi
\ifdefined\tablename
  \renewcommand*\tablename{Tabela}
\else
  \newcommand\tablename{Tabela}
\fi
}
\@ifpackageloaded{float}{}{\usepackage{float}}
\floatstyle{ruled}
\@ifundefined{c@chapter}{\newfloat{codelisting}{h}{lop}}{\newfloat{codelisting}{h}{lop}[chapter]}
\floatname{codelisting}{Listagem}
\newcommand*\listoflistings{\listof{codelisting}{Lista de Listagens}}
\makeatother
\makeatletter
\makeatother
\makeatletter
\@ifpackageloaded{caption}{}{\usepackage{caption}}
\@ifpackageloaded{subcaption}{}{\usepackage{subcaption}}
\makeatother
\ifLuaTeX
\usepackage[bidi=basic]{babel}
\else
\usepackage[bidi=default]{babel}
\fi
\babelprovide[main,import]{brazilian}
% get rid of language-specific shorthands (see #6817):
\let\LanguageShortHands\languageshorthands
\def\languageshorthands#1{}
\ifLuaTeX
  \usepackage{selnolig}  % disable illegal ligatures
\fi
\usepackage{bookmark}

\IfFileExists{xurl.sty}{\usepackage{xurl}}{} % add URL line breaks if available
\urlstyle{same} % disable monospaced font for URLs
\hypersetup{
  pdftitle={Enriquecendo o ensino de frações através da integração da matemática e música},
  pdfauthor={Alberson Miranda},
  pdflang={pt-BR},
  colorlinks=true,
  linkcolor={blue},
  filecolor={Maroon},
  citecolor={Blue},
  urlcolor={Blue},
  pdfcreator={LaTeX via pandoc}}

\title{Enriquecendo o ensino de frações através da integração da
matemática e música\thanks{Trabalho desenvolvido para a disciplina de
Tópicos Especiais em Educação Matemática, ministrada pela professora
Débora Domingues, no curso de Licenciatura em Matemática do Instituto
Federal do Espírito Santo.}}
\author{Alberson Miranda}
\date{22 de abril de 2024}

\begin{document}
\maketitle

\renewcommand*\contentsname{Índice}
{
\hypersetup{linkcolor=}
\setcounter{tocdepth}{2}
\tableofcontents
}
\chapter{INTRODUÇÃO}\label{introduuxe7uxe3o}

A música é um instrumento poderoso para promoção dessa
interdisciplinaridade, não apenas porque ela está enraizada no
imaginário de cada criança, que desde o berço escuta o canto da mãe ao
ninar, mas também por ser uma forma de arte que envolve a matemática em
sua estrutura.

\chapter{FUNDAMENTAÇÃO TEÓRICA}\label{fundamentauxe7uxe3o-teuxf3rica}

As chamadas tendências em educação matemática são categorizações que
buscam identificar e descrever as diferentes abordagens e perspectivas
que norteiam a pesquisa e o ensino da matemática (MELLO, 2007). Nesse
sentido, uma das tendências que surge a partir da busca de soluções para
os problemas da Educação Matemática é a interdisciplinaridade.

A interdisciplinaridade é uma abordagem que visa a integração de
diferentes áreas do conhecimento, com o objetivo de promover uma
aprendizagem mais significativa e contextualizada. Esse conceito envolve
ao mesmo tempo teoria e ação, uma vez que exige mais a atuação do
professor em sala de aula do que a simples união de duas ou mais
disciplinas ou áreas do saber em atividades (MELLO, 2007).

Essa interdisciplinaridade é alcançada a partir do rompimento com o
isolamento e a fragmentação dos conteúdos, possibilitando a
transferência de aprendizagem de uma situação para a outra e a
construção de significado em cima desse aprendizado transferido (SOUTO,
2010). Para que isso seja possível, SOUTO (2010) lista algumas condições
que a atividade deve atender, como:

\begin{enumerate}
\def\labelenumi{\arabic{enumi}.}
\tightlist
\item
  O tema deve ser algo conhecido dos alunos;
\item
  Ser de discussão possível;
\item
  Ter valor em si mesmo;
\item
  Ser capaz de criar conceitos matemáticos;
\item
  desenvolver habilidades matemáticas;
\item
  e privilegiar a concretude social.
\end{enumerate}

Nesse sentido, a integração com a arte é uma das formas de promover a
interdisciplinaridade, uma vez que ela é uma forma de expressão humana
que permeia o indivíduo em toda cultura e sociedade. De acordo com
ROBINSON (2013), integração com artes pode ser definida a partir de três
características que devem ser consideradas para que seja alcançada uma
interdisciplinaridade de alta qualidade, são elas:

\begin{enumerate}
\def\labelenumi{\arabic{enumi}.}
\tightlist
\item
  Aprendizado \emph{através} e \emph{com} artes;
\item
  Artes como processo de conexão curricular;
\item
  Artes como engajamento colaborativo.
\end{enumerate}

BRESLER (1995) realizou um estudo etnográfico em três escolas K-8 nos
Estados Unidos\footnote{K-8 é uma abreviação para \emph{kindergarten}
  (pré-escola) até o 8º ano do ensino fundamental.}, incluindo
observações de aulas; entrevistas com professores, diretores e artistas
residentes; e revisão de materiais curriculares. A partir desse estudo,
a autora definiu quarto abordagens de integração com a arte,
sintetizadas por ROBINSON (2013), são elas:

\begin{enumerate}
\def\labelenumi{\arabic{enumi}.}
\tightlist
\item
  \textbf{Integração subserviente}: a arte é apenas um extra, usada para
  ilustrar ou reforçar conceitos de outras disciplinas;
\item
  \textbf{Integração afetiva}: a integração se dá por meio da imersão e
  da consequente reação dos alunos à arte, como música e peças
  artísticas, complementando o currículo de outras disciplinas;
\item
  \textbf{Integração social}: baseada em atividades, utilizando a arte
  para promover a interação entre os alunos e aumentar a participação
  parental, como em peças de teatro ou música em grupo;
\item
  \textbf{Integração co-igual cognitiva}: a arte é integrada com outros
  aspectos do currículo e os alunos são exigidos a usar habilidades de
  pensamento de ordem superior e qualidades estéticas para obter um
  entendimento mais aprofundado de um conceito acadêmico específico;
\end{enumerate}

As três primeiras abordagens utilizam a arte como uma ferramenta. Já a
quarta abordagem, a integração co-igual cognitiva, é a mais exigente,
demandando do professor não apenas o conhecimento, habilidade e
confiança no seu conteúdo, mas também na forma de arte escolhida. Além
disso, requer tempo para planejar e efetivamente preparar aulas que
integrem a arte com o conteúdo acadêmico (LOVEMORE; ROBERTSON; GRAVEN,
2021).

\phantomsection\label{refs}
\begin{CSLReferences}{0}{1}
\bibitem[\citeproctext]{ref-bresler_subservient_1995}
BRESLER, L. \href{https://doi.org/10.1080/10632913.1995.9934564}{The
{Subservient}, {Co}-{Equal}, {Affective}, and {Social} {Integration}
{Styles} and their {Implications} for the {Arts}}. \textbf{Arts
Education Policy Review}, v. 96, n. 5, p. 31--37, jun. 1995.

\bibitem[\citeproctext]{ref-lovemore_enriching_2021}
LOVEMORE, T.; ROBERTSON, S.-A.; GRAVEN, M.
\href{https://doi.org/10.4102/sajce.v11i1.899}{Enriching the teaching of
fractions through integrating mathematics and music}. \textbf{South
African Journal of Childhood Education}, v. 11, jan. 2021.

\bibitem[\citeproctext]{ref-mello_tendencias_2007}
MELLO, A. C. C. D. \textbf{Tendências {Em} {Educação} {Matemática}}.
{[}s.l.{]} Unisulvirtual, 2007.

\bibitem[\citeproctext]{ref-robinson_arts_2013}
ROBINSON, A. H. \href{https://doi.org/10.1080/10632913.2013.826050}{Arts
{Integration} and the {Success} of {Disadvantaged} {Students}: {A}
{Research} {Evaluation}}. \textbf{Arts Education Policy Review}, v. 114,
n. 4, p. 191--204, out. 2013.

\bibitem[\citeproctext]{ref-souto_interdisciplinaridade_2010}
SOUTO, D. L. P. Interdisciplinaridade e aprendizagem da {Matemática} em
sala de aula, de {Vanessa} {Sena} {Tomaz} e {Maria} {Manuela} {Martins}
{Soares} {David}.({Coleção} {Tendências} em {Educação}
{Matemática})--{Belo} {Horizonte}: {Autêntica}, 2008.
\textbf{Bolema-Boletim de Educação Matemática}, v. 23, n. 36, p.
801--808, 2010.

\end{CSLReferences}



\end{document}
